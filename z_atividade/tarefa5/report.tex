\documentclass[12pt]{article}
\usepackage{fontspec}
\usepackage[brazilian]{babel}
\usepackage{listings}
\usepackage{xcolor}
\usepackage{geometry}

\geometry{
  a4paper,
  margin=2.5cm
}

\lstset{
  basicstyle=\small\ttfamily,
  keywordstyle=\color{blue},
  commentstyle=\color{green!50!black},
  stringstyle=\color{orange},
  breaklines=true,
  frame=single,
  numbers=left,
  numberstyle=\tiny\color{gray},
  numbersep=5pt,
  showstringspaces=false,
  tabsize=2
}

\title{Tarefa 5: Concorrência com threads}
\author{}
\date{\today}

\begin{document}

\maketitle

\section{Implementação}

Foi implementado um programa em linguagem C que cria duas threads que incrementam concorrentemente uma variável global compartilhada:

\begin{lstlisting}[language=C]
#include <stdio.h>
#include <stdint.h>
#include <pthread.h>

uint64_t valor = 0;

void* thread(void* arg) {
    size_t i = 1000000;
    while (i--) {
        valor++;
    }
    return NULL;
}

int main() {
    pthread_t t1, t2;

    // Criar duas threads
    pthread_create(&t1, NULL, thread, NULL);
    pthread_create(&t2, NULL, thread, NULL);

    // Aguardar as threads terminarem
    pthread_join(t1, NULL);
    pthread_join(t2, NULL);

    // Imprimir o resultado
    printf("Valor final: %lu\n", valor);

    return 0;
}
\end{lstlisting}

O programa foi compilado de duas formas diferentes: uma com as configurações padrão do GCC e outra com otimizações habilitadas (\texttt{-O3}).

\section{Análise do Comportamento}

\subsection{Resultado Sem Otimizações}

Ao executar o programa compilado com as configurações padrão do GCC, o resultado final é consistentemente menor que 2 milhões, variando em cada execução.

\subsection{Resultado Com Otimizações}

Ao compilar o programa com a flag \texttt{-O3}, o resultado é sempre exatamente 2 milhões.

\section{Respostas às Questões}

\subsection{Por que o resultado não é 2 milhões quando compilado com gcc nas configurações padrões?}

O resultado não é 2 milhões devido a uma condição de corrida (\textit{race condition}) entre as duas threads. A operação \texttt{valor++} não é atômica e, na realidade, é composta por três operações distintas:

\begin{enumerate}
    \item Ler o valor atual da variável da memória para um registrador
    \item Incrementar o valor no registrador
    \item Escrever o novo valor de volta na memória
\end{enumerate}

Sem otimizações, as duas threads executam estas três operações concorrentemente, o que pode levar a cenários como:

\begin{enumerate}
    \item Thread 1 lê o valor atual (por exemplo, 1000)
    \item Thread 2 lê o mesmo valor atual (1000)
    \item Thread 1 incrementa para 1001
    \item Thread 2 incrementa para 1001
    \item Thread 1 escreve 1001
    \item Thread 2 escreve 1001
\end{enumerate}

Neste cenário, embora duas operações de incremento tenham sido realizadas, o valor final é incrementado apenas uma vez. Este fenômeno, conhecido como condição de corrida, ocorre repetidamente durante a execução, resultando em um valor final menor que 2 milhões.

\subsection{Por que o resultado é 2 milhões quando habilitamos as otimizações do gcc?}

Com as otimizações habilitadas (\texttt{-O3}), o compilador GCC transforma significativamente o código gerado. Ao analisar o código assembly produzido, observamos que o compilador realiza otimizações agressivas:

\begin{enumerate}
    \item O compilador percebe que o laço \texttt{while (i--)} está simplesmente incrementando \texttt{valor} um milhão de vezes
    \item Em vez de executar o incremento em um laço, o compilador substitui todo o código da função por uma única operação que adiciona 1.000.000 diretamente à variável \texttt{valor}
    \item Esta única operação é atômica do ponto de vista da função, resultando em exatamente 2 milhões após ambas as threads serem executadas
\end{enumerate}

Essencialmente, com a otimização, cada thread efetivamente executa \texttt{valor += 1000000} como uma operação única, eliminando o problema de concorrência presente na versão sem otimização.

\subsection{Com as otimizações habilitadas, seria possível o valor final ser menos de 2 milhões? Por quê?}

Sim, ainda seria teoricamente possível obter um valor menor que 2 milhões mesmo com as otimizações habilitadas. Embora o compilador tenha otimizado o código para que cada thread realize uma única operação \texttt{valor += 1000000}, esta operação ainda não é atômica no nível do hardware quando se trata de múltiplas threads.

A operação otimizada ainda segue o mesmo padrão básico de leitura-modificação-escrita:

\begin{enumerate}
    \item Ler o valor atual de \texttt{valor}
    \item Adicionar 1.000.000
    \item Escrever o resultado de volta
\end{enumerate}

Se a segunda thread ler o valor inicial antes que a primeira thread tenha escrito seu resultado, ainda ocorrerá uma condição de corrida. No entanto, a probabilidade disso acontecer é drasticamente reduzida em comparação com a versão sem otimização, pois agora há apenas uma operação por thread em vez de um milhão.

Na prática, devido ao tempo necessário para a criação das threads e a baixa probabilidade de colisão exata no momento crítico, o resultado quase sempre será 2 milhões com otimizações, mas não há garantia absoluta.

\subsection{Forma de implementar que resolve o problema da contagem errada}

Uma solução para este problema seria utilizar mecanismos de sincronização, como mutexes ou operações atômicas:

\begin{lstlisting}[language=C]
#include <stdio.h>
#include <stdint.h>
#include <pthread.h>

uint64_t valor = 0;
pthread_mutex_t mutex = PTHREAD_MUTEX_INITIALIZER;

void* thread(void* arg) {
    size_t i = 1000000;
    while (i--) {
        pthread_mutex_lock(&mutex);
        valor++;
        pthread_mutex_unlock(&mutex);
    }
    return NULL;
}
\end{lstlisting}

Alternativa usando operações atômicas:

\begin{lstlisting}[language=C]
#include <stdio.h>
#include <stdint.h>
#include <pthread.h>
#include <stdatomic.h>

atomic_uint_least64_t valor = 0;

void* thread(void* arg) {
    size_t i = 1000000;
    while (i--) {
        atomic_fetch_add(&valor, 1);
    }
    return NULL;
}
\end{lstlisting}

Estas soluções resolvem o problema porque garantem que as operações de incremento sejam atômicas. No caso do mutex, apenas uma thread por vez pode executar o código protegido pelo mutex, eliminando a condição de corrida. As operações atômicas, por sua vez, utilizam instruções especiais de hardware que garantem atomicidade para operações básicas como incremento, também eliminando a condição de corrida.

Ambas as soluções garantirão que o resultado seja consistentemente 2 milhões, independentemente das otimizações do compilador. A solução com operações atômicas tende a ser mais eficiente que a solução com mutex, especialmente para operações simples como incremento.

\section{Conclusão}

Este trabalho demonstra os desafios intrínsecos à programação concorrente e como o comportamento de programas com threads pode ser significativamente afetado pelas otimizações do compilador. A condição de corrida observada é um problema clássico em sistemas concorrentes, e sua resolução requer o uso adequado de mecanismos de sincronização.

Além disso, este exemplo ilustra como as otimizações do compilador podem modificar drasticamente o comportamento do programa, às vezes mascarando problemas subjacentes de concorrência que permanecem teoricamente presentes, mesmo que raramente se manifestem na prática.

Para garantir a correção de programas concorrentes, é essencial compreender os mecanismos de sincronização disponíveis e aplicá-los adequadamente, independentemente do nível de otimização utilizado na compilação.

\end{document}
